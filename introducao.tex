 \section{Introdução}
 
\IEEEPARstart{C}{ontroladores} eletrônicos de estabilidade
(\emph{ESC}) são sistemas de assistência ao motorista na correção de movimentos que desviem o veículo da
trajetória desejada, contribuindo para a redução de acidentes causados
pelo deslizamento lateral e capotamento. 
Muitas pesquisas foram realizadas nos últimos anos para propor implemetações de
\emph{ESCs}, em que várias técnicas de controle têm sido aplicadas
\cite{Zhang2016, Zhai2016, Yuan2016,Li2017, Nahidi2017, Jin2017a, Jalali2017, Niaona2017, Guo2017}.
Em \cite{Yogurtcu2015} foram desenvolvidos dois sistemas que aplicam o
\emph{PID} e o \emph{LQR}, simulações do controle do veículo representado pelo
modelo bicicleta foram realizadas para validar estes controladores.
Em \cite{Subroto2017} um controle de modo deslizante adaptativo foi projetado
para aumentar a estabilidade lateral de veículos com tração independente nas quatro rodas.
Em \cite{Zhou2010} a técnica \emph{Backstepping} foi aplicada no projeto
de um controlador da taxa de guinada mediante comando do sistema de freios.

Uma arquitetura comum nestes sistemas é a separação do controle em dois níveis:
o alto nível, que calcula as forças e torques que precisam ser aplicados ao
veículo para estabilizá-lo, e o baixo nível, que
comanda os atuadores para que os valores calculados pelo alto
nível sejam alcançados.
Uma vantagem dessa arquitetura é o potencial
do controlador de alto nível para ser aplicado a veículos com diferentes
sistemas de atuação, mas com a desvatangem de não ser auto suficiente por
depender de um controlador de baixo nível para comandar os atuadores.

Este artigo apresenta um controlador de alto nível que aplica o
controle preditivo baseado em modelo linear sem restrições. Este
controlador foi implementado como uma preparação para um controle preditivo
baseado em modelo não linear que considere as restrições físicas do sinal
de comando e imponha restrições aos estados para mantê-los dentro
das faixas de valores desejados.  

Em \cite{Li2017} e \cite{Nahidi2017} foram propostos
controladores preditivos (\emph{MPC}) para calcular a força exigida de cada
roda, e para a obtenção dessas forças, foi utilizado em \cite{Li2017} um
controlador robusto do sistema de freios diferenciais e em  \cite{Nahidi2017} um
controlador ótimo do sistema de distribuição de torque.
O controlador proposto neste artigo calcula o momento de guinada externo a ser
aplicado ao veículo para corrigir o deslizamento lateral e movimentos de
rolagem indesejados e fazer a taxa de guinada rastrear uma trajetória desejada.

Neste trabalho, o algoritmo de controle foi implementado como uma aplicação para
Linux. Para validar está implementação foram realizadas simulações em que
o algoritmo de controle e o modelo da dinâmica veícula são executados em uma
\emph{RaspberryPi C 2010.2011} com sistema operacional 
\emph{RASPBIAN STRETCH LITE version Appril 2018}.

Este documento está organizado da seguinte forma: a Seção
\ref{sec:model} apresenta o modelo matemático utilizado para simular a dinâmica
veicular; a Seção \ref{sec:control} descreve o sistema de controle proposto;
a Seção \ref{sec:result} apresenta os experimentos de validação do sistema
proposto e a Seção \ref{sec:conc} apresenta a conclusão.