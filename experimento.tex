 \section{ Dinâmica veicular }
 \label{sec:model}
 
O modelo da dinâmica veicular utilizado neste trabalho, a partir do qual
se obteve o modelo linear utilizado para projetar o controlador e simular
a dinâmica veícular, é composto dos movimentos lateral, longitudinal e de
guinada e de rolagem \cite{Zheng2006,Miao2015}.
Como pretende-se observar a resposta do veículo ao momento de guinada externo
calculado pelo controlador sem considerar o sistema de atuação utilizado para
isso, o modelo não considera as forças longitudinais geradas pelas rodas, que
dependem do efeito dos atuadores na tração transferida para as rodas
\cite{Shoutao2017}.
A Fig. \ref{fig:modeldraw}  ilustra este modelo represetentado pelas seguintes
equações:

\begin{figure}[t]
    \centering
    \includegraphics[width=\columnwidth]{carmodel.png}
    \caption{ Modelo do veículo com as forças aplicadas a ela e suas componentes
    de movimento}
    \label{fig:modeldraw}
\end{figure}

\noindent\textbf{Movimento longitudinal:}
    \begin{equation}
    \label{eq:model}
    \begin{split}
        m\left(\dot{u} - \dot{\psi} v \right) + m_sh_s\dot{\psi}\dot{\phi} = 
    	-\sin(\delta_f)( F_{yfl} + F_{yfr} )  
    	\\
    \end{split}
    \end{equation}
\noindent\textbf{Movimento lateral:} 
    \begin{equation}
    \label{eq:model}
    \begin{split}
    	m\left(\dot{v} + \dot{\psi} u \right) - m_sh_s\ddot{\phi} =& 
    	\cos(\delta_f)( F_{yfl} + F_{yfr} )\\ +&  F_{yrl} +  F_{yrr} \\
    \end{split}
    \end{equation}
\noindent\textbf{Movimento de guinada:} 
    \begin{equation}
    \label{eq:model}
    \begin{split}
        I_{zz}\ddot{\psi} - I_{xz}\ddot{\phi} = 
          &   a\cos(\delta_f)(F_{yfl} + F_{yfr}) - b(F_{yrl} + F_{yrr}) \\
        - & \frac{t\sin(\delta_f)}{2}(F_{yfr} - F_{yfl} ) \\
        + & M_{u}
    \end{split}
    \end{equation}
\noindent\textbf{Movimento de rolagem:}
\begin{equation}
\label{eq:model}
\begin{split}
    I_{xx}\ddot{\phi} - I_{xz}\ddot{\psi} & = m_sh_s(\dot{v} +
    \dot{\psi}u) + m_sh_sg\sin\phi \\
    &- (k_{\phi f} +  K_{\phi r})\phi - (c_{\phi f} + c_{\phi r})\dot{\phi} \\
 \end{split}        
\end{equation}

\noindent em que $v$ é a velocidade lateral, $u$ é a velocidade longitudinal,
$\psi$ é o ângulo de guinada, $\phi$ é o ângulo de rolagem, $\delta_f$ é o
ângulo de esterçamento das rodas dianteiras, $m$ é a massa total do veículo, 
$m_s$ é a massa do veículo sobre o eixo de rolagem, $h_s$ é a distância entre o
centro de rolagem e o centro de massa, $I_{zz}$ e $I_{xx}$ são os momentos de inércia de guinada e rolagem, respectivamente,
$I_{xz}$ é o produto de inércia em relação aos eixos de guinada e de rolagem,
$M_u$ é o momento externo de guinada calculado e
$F_{yfl},F_{yr},F_{yrl},F_{yrr}$ são as forças latereais exercidas pelas rodas.

\vspace{-5mm}
\subsection{ Dinâmica dos pneus }

As forças geradas pelas rodas são governadas por relações complexas entre o
atrito pneu-pista, carga, ângulo de deslizamento e propriedades dos pneus. Os
modelos amplamente utilizados baseiam-se em formulações empíricas derivadas de
testes experimentais \cite{Dugoff1969}. 
As não linearidades na dinâmica dos pneus 
foram consideradas neste trabalho pela utilização da \emph{Magic Formula} (\emph{MF}) 
para calcular a força exercida por cada roda \cite{Shoutao2017}.
Na \emph{MF} a força lateral $F_{yi}$ exercida pela roda é modelada em função de
carga $F_{zi}$ e o ângulo de deslizamento lateral $\alpha_i$ da roda.
A formulação geral da \emph{MF} é expressa por \cite{Li2017}:
\vspace{-2.5mm}
\begin{equation}
\label{eq:MFY}
\begin{split}
		F_{wyi}  & = D\sin(C\arctan[B(1-E)\alpha_i \\ 
			     & + E\arctan(B\alpha_i)]) \\
			  C  & = a_0 \\
			  D  & = F_{zi}(a_1F_{zi} + a_2 ) \\
			  B  & = \frac{(a_3\sin(2\arctan\frac{F_{zi}}{a_4})}{CD} \\
			  E  & = a_6F_{zi} + a_7 \\
			  i  & = fl,fr,rl,rr
\end{split}
\end{equation}

\noindent em que $\gamma$ é o ângulo entre o eixo de rotação da roda e a pista,
normalmente assumido como zero 

Considerando a aceleração longitudinal $a_x$ e lateral $a_y$, a carga individual
de cada roda pode ser calculada por:
\begin{equation}
\label{eq:wload}
\begin{split}
    F_{zfl} =  \frac{mgb}{2l} - \frac{ma_xh}{2l} - \frac{ma_yah}{lt_f} -
    \frac{k_{\phi f}\phi}{t_f} - \frac{c_{\phi f}\dot\phi}{t_f} \\
    F_{zfr} =  \frac{mgb}{2l} - \frac{ma_xh}{2l} + \frac{ma_yah}{lt_f} +
    \frac{k_{\phi f}\phi}{t_f} + \frac{c_{\phi f}\dot\phi}{t_f} \\
    F_{zrl} =  \frac{mga}{2l} + \frac{ma_xh}{2l} - \frac{ma_yah}{lt_r} - 
    \frac{k_{\phi r}\phi}{t_r} - \frac{c_{\phi r}\dot\phi}{t_r} \\
    F_{zrr} =  \frac{mga}{2l} + \frac{ma_xh}{2l} + \frac{ma_yah}{lt_r} + 
    \frac{k_{\phi r}\phi}{t_r} + \frac{c_{\phi r}\dot\phi}{t_r}
\end{split}
\end{equation}
\noindent em que $a$ é a distância das rodas dianteiras do centro de massa, $b$
é a distância das rodas traseiras do centro de massa, $h$ é a altura do veículo,
$l$ é a distância entre os eixos dianteiro e traseiro, $t_f$ e $t_r$ são as
larguras dos eixos dianteira e traseiro, respectivamente.

O ângulo de deslizamento lateral de cada roda, definido como o ângulo entre o
vetor velocidade linear da roda e o seu eixo longitudinal, pode ser calculado
por:
\begin{equation}
	\label{eq:alphaall}
	\begin{split}
		\alpha_{fl} = \delta_f - \arctan\left(\frac{v + a\dot{\psi}}{u -
		\frac{t_f}{2}\dot{\psi}}\right)  & \enskip \alpha_{rl} = -
		\arctan\left(\frac{v - b\dot{\psi}}{u - \frac{t_r}{2}\dot{\psi}}\right)\\
		\alpha_{fr}  = \delta_f - \arctan\left(\frac{v + a\dot{\psi}}{u +
		\frac{t_f}{2}\dot{\psi}}\right)  & \enskip  \alpha_{rr} = - \arctan\left(\frac{v - b\dot{\psi}}{u +
        \frac{t_r}{2}\dot{\psi}}\right)\\
	\end{split}
\end{equation}

\section{Sistema de controle}
\label{sec:control}
    
O sistema proposto aplica o controle preditivo baseado em modelo linear
sem restrições para calcular o momento de guinada externo em função dos erros
da taxa de guinada, do ângulo de deslizamento lateral, da taxa de rolagem
e do ângulo de rolagem.

Considera-se que o controlador tem acesso aos quatro estados do modelo linear.
Como os movimentos de rolagem e o deslizamento lateral são movimentos
indesejados, a trajetória desejada para estes estados é constante e igual a
zero.
O valor de referência assumido para a taxa de guinada é calculado conforme 
descrito em \cite{Jin2017,Zheng2006}:
\begin{equation}
	\label{eq:desiredyaw}
	\begin{split}
		\dot{\psi}_d  = \frac{u\delta_f}{l+lK_uu^2} \enskip&\enskip 
		K_u = \frac{m}{l^2}\left(\frac{b}{C_r} - \frac{a}{C_f}\right)
	\end{split}
\end{equation}
\begin{equation}
	\begin{split}
		|\dot{\psi}_d| \leq \left|\frac{\mu g}{u}\right| 
	\end{split}
\end{equation}

O  momento de guinada é calculado pela lei de controle :

$M_u(k)= K_Nx(k) + G_N\tilde{y}_d + L_Nu_d $,
\noindent em que as matrizes matrizes $K_N$, $G_N$, $L_N$ são obtidos
conforme descrito em \cite{alamir2013pragmatic} como solução ótima da função
custo sem restrição do controle preditivo para sistemas LTI, $x$ é o vetor de
estados, $\tilde{y}_d$ é um vetor composto da concatenação dos vetores de
estados que formam a trajetória desejada no hirizonte de predição e $u_d$ é 
o vetor de comando desejado para o estado estacionário, assumido como zero
porque considera-se que as curvas do veículo são transitórias, pois o veículo
sempre retorna para a condução em linha reta quando nenhum momomento externo
deve ser aplicado.

O modelo LTI utilizado no projeto do \emph{MPC}} consiste de uma
linearização do modelo apresentado para a dinâmica veicular, assumindo
velocidade longitudinal constante igual a 80km/h e desprezando a equação
do movimento longitudinal. O resultado é um modelo composto pelos movimentos
lateral, de guinada e de rolagem representados pelas equações:
\textbf{Movimento lateral:}
\begin{equation}
\label{eq:linearlat}
\begin{split}
	m\left(\dot{v} + \dot{\psi} u \right) -m_sh_s\ddot{\phi}= \sum{F_y} \\
\end{split}
\end{equation}
\textbf{Movimento de guinada:} 
\begin{equation}
\label{eq:lineargui}
\begin{split}
	I_{zz}\ddot{\psi} - I_{xz}\ddot{\phi} = a(F_{yfl} + F_{yfr}) - b(F_{yrl} +
	F_{rr}) + M_u \\
\end{split}
\end{equation}
\textbf{Movimento de rolagem:}
\begin{equation}
\label{eq:linearrol}
\begin{split}
	I_{xx}\ddot{\phi} - I_{xz}\ddot{\psi} = & m_sh_s(\dot{v} + \dot{\psi}u) +
	m_sh_sg\phi - \\&(k_{\phi f} + k_{\phi r} ) \phi - ( c_{\phi f} + c_{\phi r} )
	\dot{\phi} \\
\end{split}
\end{equation}

Para as forças dos pneus é utilizada a aproximação linear da \emph{MF}:
\begin{equation}
	\label{eq:FMF}
\begin{split}
	F_{yfi} \approx C_f\alpha_f \enskip&\enskip F_{yri} \approx C_r\alpha_r
\end{split}
\end{equation} 
\noindent em que $C_r$ e $C_f$ são denominados coeficientes de rigidez nas curvas das rodas traseiras e dianteiras, 
respectivamente, os seus valores são obtidos por expansão de Taylor:
\begin{equation}
\begin{split}
		C_f = \left.\frac{\partial F_{yi}}{\partial \alpha_i}\right|_{x_e} \enskip&\enskip
		C_r = \left.\frac{\partial F_{yi}}{\partial \alpha_i}\right|_{x_e}\\
	i = fl,fr \enskip&\enskip i = rl,rr
	\end{split}
\end{equation}
\noindent em que $x_e$ representa o ponto de linearização.

O ângulo de deslizamento das rodas aproximado é calculado em função do deslizamento lateral $\beta$, 
da taxa de rolagem $\dot{\psi}$ e do esterçamento das rodas dianteiras $\delta_f$:
\begin{equation}
	\label{eq:alphalin}
\begin{split}
		\alpha_f  = -\beta - \frac{a\dot{\psi}}{u} + \delta_f \enskip&\enskip
		\alpha_r  = -\beta + \frac{b\dot{\psi}}{u}
	\end{split}
\end{equation}
\begin{equation}
	\label{eq:beta}
	\beta =\arctan\left(\frac{v}{u}\right)
\end{equation}	

Considerando a velocidade longitudinal constante, a manipulação algébrica das
Equações \ref{eq:linearlat} - \ref{eq:alphalin} permite obter uma representação
do modelo linear em espaço de estados: $\dot{x} = Ax + BM_u+ E\delta_f$.