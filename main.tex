

%% bare_jrnl.tex
%% V1.4b
%% 2015/08/26
%% by Michael Shell
%% see http://www.michaelshell.org/
%% for current contact information.
%%
%% This is a skeleton file demonstrating the use of IEEEtran.cls
%% (requires IEEEtran.cls version 1.8b or later) with an IEEE
%% journal paper.
%%
%% Support sites:
%% http://www.michaelshell.org/tex/ieeetran/
%% http://www.ctan.org/pkg/ieeetran
%% and
%% http://www.ieee.org/

%%*************************************************************************
%% Legal Notice:
%% This code is offered as-is without any warranty either expressed or
%% implied; without even the implied warranty of MERCHANTABILITY or
%% FITNESS FOR A PARTICULAR PURPOSE! 
%% User assumes all risk.
%% In no event shall the IEEE or any contributor to this code be liable for
%% any damages or losses, including, but not limited to, incidental,
%% consequential, or any other damages, resulting from the use or misuse
%% of any information contained here.
%%
%% All comments are the opinions of their respective authors and are not
%% necessarily endorsed by the IEEE.
%%
%% This work is distributed under the LaTeX Project Public License (LPPL)
%% ( http://www.latex-project.org/ ) version 1.3, and may be freely used,
%% distributed and modified. A copy of the LPPL, version 1.3, is included
%% in the base LaTeX documentation of all distributions of LaTeX released
%% 2003/12/01 or later.
%% Retain all contribution notices and credits.
%% ** Modified files should be clearly indicated as such, including  **
%% ** renaming them and changing author support contact information. **
%%*************************************************************************


% *** Authors should verify (and, if needed, correct) their LaTeX system  ***
% *** with the testflow diagnostic prior to trusting their LaTeX platform ***
% *** with production work. The IEEE's font choices and paper sizes can   ***
% *** trigger bugs that do not appear when using other class files.       ***                          ***
% The testflow support page is at:
% http://www.michaelshell.org/tex/testflow/
	
\documentclass[journal,letterpaper]{IEEEtran}
\makeatletter
\renewcommand\footnoterule{%
  \kern-3\p@
  \hrule\@width.4\columnwidth
  \kern2.6\p@}
  \makeatother


\pagestyle{empty}
  \usepackage{graphicx}
  \usepackage[]{subfigure}
%  \usepackage[portuguese,algoruled,longend]{algorithm2e}
  \usepackage{multirow}
  \usepackage{hyphenat}
  \usepackage{cite}
  \usepackage{adjustbox}
  \usepackage{subfig}
  \usepackage{float}
  \usepackage{amsmath} 
  \usepackage[brazilian,english]{babel}
  \usepackage[utf8]{inputenc}
  \usepackage[T1]{fontenc}  
%
% If IEEEtran.cls has not been installed into the LaTeX system files,
% manually specify the path to it like:
% \documentclass[journal]{../sty/IEEEtran}





% Some very useful LaTeX packages include:
% (uncomment the ones you want to load)


% *** MISC UTILITY PACKAGES ***
%
%\usepackage{ifpdf}
% Heiko Oberdiek's ifpdf.sty is very useful if you need conditional
% compilation based on whether the output is pdf or dvi.
% usage:
% \ifpdf
%   % pdf code
% \else
%   % dvi code
% \fi
% The latest version of ifpdf.sty can be obtained from:
% http://www.ctan.org/pkg/ifpdf
% Also, note that IEEEtran.cls V1.7 and later provides a builtin
% \ifCLASSINFOpdf conditional that works the same way.
% When switching from latex to pdflatex and vice-versa, the compiler may
% have to be run twice to clear warning/error messages.






% *** CITATION PACKAGES ***
%
%\usepackage{cite}
% cite.sty was written by Donald Arseneau
% V1.6 and later of IEEEtran pre-defines the format of the cite.sty package
% \cite{} output to follow that of the IEEE. Loading the cite package will
% result in citation numbers being automatically sorted and properly
% "compressed/ranged". e.g., [1], [9], [2], [7], [5], [6] without using
% cite.sty will become [1], [2], [5]--[7], [9] using cite.sty. cite.sty's
% \cite will automatically add leading space, if needed. Use cite.sty's
% noadjust option (cite.sty V3.8 and later) if you want to turn this off
% such as if a citation ever needs to be enclosed in parenthesis.
% cite.sty is already installed on most LaTeX systems. Be sure and use
% version 5.0 (2009-03-20) and later if using hyperref.sty.
% The latest version can be obtained at:
% http://www.ctan.org/pkg/cite
% The documentation is contained in the cite.sty file itself.






% *** GRAPHICS RELATED PACKAGES ***
%
\ifCLASSINFOpdf
  % \usepackage[pdftex]{graphicx}
  % declare the path(s) where your graphic files are
  % \graphicspath{{../pdf/}{../jpeg/}}
  % and their extensions so you won't have to specify these with
  % every instance of \includegraphics
  % \DeclareGraphicsExtensions{.pdf,.jpeg,.png}
\else
  % or other class option (dvipsone, dvipdf, if not using dvips). graphicx
  % will default to the driver specified in the system graphics.cfg if no
  % driver is specified.
  % \usepackage[dvips]{graphicx}
  % declare the path(s) where your graphic files are
  % \graphicspath{{../eps/}}
  % and their extensions so you won't have to specify these with
  % every instance of \includegraphics
  % \DeclareGraphicsExtensions{.eps}
\fi
% graphicx was written by David Carlisle and Sebastian Rahtz. It is
% required if you want graphics, photos, etc. graphicx.sty is already
% installed on most LaTeX systems. The latest version and documentation
% can be obtained at: 
% http://www.ctan.org/pkg/graphicx
% Another good source of documentation is "Using Imported Graphics in
% LaTeX2e" by Keith Reckdahl which can be found at:
% http://www.ctan.org/pkg/epslatex
%
% latex, and pdflatex in dvi mode, support graphics in encapsulated
% postscript (.eps) format. pdflatex in pdf mode supports graphics
% in .pdf, .jpeg, .png and .mps (metapost) formats. Users should ensure
% that all non-photo figures use a vector format (.eps, .pdf, .mps) and
% not a bitmapped formats (.jpeg, .png). The IEEE frowns on bitmapped formats
% which can result in "jaggedy"/blurry rendering of lines and letters as
% well as large increases in file sizes.
%
% You can find documentation about the pdfTeX application at:
% http://www.tug.org/applications/pdftex





% *** MATH PACKAGES ***
%
%\usepackage{amsmath}
% A popular package from the American Mathematical Society that provides
% many useful and powerful commands for dealing with mathematics.
%
% Note that the amsmath package sets \interdisplaylinepenalty to 10000
% thus preventing page breaks from occurring within multiline equations. Use:
%\interdisplaylinepenalty=2500
% after loading amsmath to restore such page breaks as IEEEtran.cls normally
% does. amsmath.sty is already installed on most LaTeX systems. The latest
% version and documentation can be obtained at:
% http://www.ctan.org/pkg/amsmath





% *** SPECIALIZED LIST PACKAGES ***
%
%\usepackage{algorithmic}
% algorithmic.sty was written by Peter Williams and Rogerio Brito.
% This package provides an algorithmic environment fo describing algorithms.
% You can use the algorithmic environment in-text or within a figure
% environment to provide for a floating algorithm. Do NOT use the algorithm
% floating environment provided by algorithm.sty (by the same authors) or
% algorithm2e.sty (by Christophe Fiorio) as the IEEE does not use dedicated
% algorithm float types and packages that provide these will not provide
% correct IEEE style captions. The latest version and documentation of
% algorithmic.sty can be obtained at:
% http://www.ctan.org/pkg/algorithms
% Also of interest may be the (relatively newer and more customizable)
% algorithmicx.sty package by Szasz Janos:
% http://www.ctan.org/pkg/algorithmicx




% *** ALIGNMENT PACKAGES ***
%
%\usepackage{array}
% Frank Mittelbach's and David Carlisle's array.sty patches and improves
% the standard LaTeX2e array and tabular environments to provide better
% appearance and additional user controls. As the default LaTeX2e table
% generation code is lacking to the point of almost being broken with
% respect to the quality of the end results, all users are strongly
% advised to use an enhanced (at the very least that provided by array.sty)
% set of table tools. array.sty is already installed on most systems. The
% latest version and documentation can be obtained at:
% http://www.ctan.org/pkg/array


% IEEEtran contains the IEEEeqnarray family of commands that can be used to
% generate multiline equations as well as matrices, tables, etc., of high
% quality.




% *** SUBFIGURE PACKAGES ***
%\ifCLASSOPTIONcompsoc
%  \usepackage[caption=false,font=normalsize,labelfont=sf,textfont=sf]{subfig}
%\else
%  \usepackage[caption=false,font=footnotesize]{subfig}
%\fi
% subfig.sty, written by Steven Douglas Cochran, is the modern replacement
% for subfigure.sty, the latter of which is no longer maintained and is
% incompatible with some LaTeX packages including fixltx2e. However,
% subfig.sty requires and automatically loads Axel Sommerfeldt's caption.sty
% which will override IEEEtran.cls' handling of captions and this will result
% in non-IEEE style figure/table captions. To prevent this problem, be sure
% and invoke subfig.sty's "caption=false" package option (available since
% subfig.sty version 1.3, 2005/06/28) as this is will preserve IEEEtran.cls
% handling of captions.
% Note that the Computer Society format requires a larger sans serif font
% than the serif footnote size font used in traditional IEEE formatting
% and thus the need to invoke different subfig.sty package options depending
% on whether compsoc mode has been enabled.
%
% The latest version and documentation of subfig.sty can be obtained at:
% http://www.ctan.org/pkg/subfig




% *** FLOAT PACKAGES ***
%
%\usepackage{fixltx2e}
% fixltx2e, the successor to the earlier fix2col.sty, was written by
% Frank Mittelbach and David Carlisle. This package corrects a few problems
% in the LaTeX2e kernel, the most notable of which is that in current
% LaTeX2e releases, the ordering of single and double column floats is not
% guaranteed to be preserved. Thus, an unpatched LaTeX2e can allow a
% single column figure to be placed prior to an earlier double column
% figure.
% Be aware that LaTeX2e kernels dated 2015 and later have fixltx2e.sty's
% corrections already built into the system in which case a warning will
% be issued if an attempt is made to load fixltx2e.sty as it is no longer
% needed.
% The latest version and documentation can be found at:
% http://www.ctan.org/pkg/fixltx2e


%\usepackage{stfloats}
% stfloats.sty was written by Sigitas Tolusis. This package gives LaTeX2e
% the ability to do double column floats at the bottom of the page as well
% as the top. (e.g., "\begin{figure*}[!b]" is not normally possible in
% LaTeX2e). It also provides a command:
%\fnbelowfloat
% to enable the placement of footnotes below bottom floats (the standard
% LaTeX2e kernel puts them above bottom floats). This is an invasive package
% which rewrites many portions of the LaTeX2e float routines. It may not work
% with other packages that modify the LaTeX2e float routines. The latest
% version and documentation can be obtained at:
% http://www.ctan.org/pkg/stfloats
% Do not use the stfloats baselinefloat ability as the IEEE does not allow
% \baselineskip to stretch. Authors submitting work to the IEEE should note
% that the IEEE rarely uses double column equations and that authors should try
% to avoid such use. Do not be tempted to use the cuted.sty or midfloat.sty
% packages (also by Sigitas Tolusis) as the IEEE does not format its papers in
% such ways.
% Do not attempt to use stfloats with fixltx2e as they are incompatible.
% Instead, use Morten Hogholm'a dblfloatfix which combines the features
% of both fixltx2e and stfloats:
%
% \usepackage{dblfloatfix}
% The latest version can be found at:
% http://www.ctan.org/pkg/dblfloatfix




%\ifCLASSOPTIONcaptionsoff
%  \usepackage[nomarkers]{endfloat}
% \let\MYoriglatexcaption\caption
% \renewcommand{\caption}[2][\relax]{\MYoriglatexcaption[#2]{#2}}
%\fi
% endfloat.sty was written by James Darrell McCauley, Jeff Goldberg and 
% Axel Sommerfeldt. This package may be useful when used in conjunction with 
% IEEEtran.cls'  captionsoff option. Some IEEE journals/societies require that
% submissions have lists of figures/tables at the end of the paper and that
% figures/tables without any captions are placed on a page by themselves at
% the end of the document. If needed, the draftcls IEEEtran class option or
% \CLASSINPUTbaselinestretch interface can be used to increase the line
% spacing as well. Be sure and use the nomarkers option of endfloat to
% prevent endfloat from "marking" where the figures would have been placed
% in the text. The two hack lines of code above are a slight modification of
% that suggested by in the endfloat docs (section 8.4.1) to ensure that
% the full captions always appear in the list of figures/tables - even if
% the user used the short optional argument of \caption[]{}.
% IEEE papers do not typically make use of \caption[]'s optional argument,
% so this should not be an issue. A similar trick can be used to disable
% captions of packages such as subfig.sty that lack options to turn off
% the subcaptions:
% For subfig.sty:
% \let\MYorigsubfloat\subfloat
% \renewcommand{\subfloat}[2][\relax]{\MYorigsubfloat[]{#2}}
% However, the above trick will not work if both optional arguments of
% the \subfloat command are used. Furthermore, there needs to be a
% description of each subfigure *somewhere* and endfloat does not add
% subfigure captions to its list of figures. Thus, the best approach is to
% avoid the use of subfigure captions (many IEEE journals avoid them anyway)
% and instead reference/explain all the subfigures within the main caption.
% The latest version of endfloat.sty and its documentation can obtained at:
% http://www.ctan.org/pkg/endfloat
%
% The IEEEtran \ifCLASSOPTIONcaptionsoff conditional can also be used
% later in the document, say, to conditionally put the References on a 
% page by themselves.




% *** PDF, URL AND HYPERLINK PACKAGES ***
%
%\usepackage{url}
% url.sty was written by Donald Arseneau. It provides better support for
% handling and breaking URLs. url.sty is already installed on most LaTeX
% systems. The latest version and documentation can be obtained at:
% http://www.ctan.org/pkg/url
% Basically, \url{my_url_here}.




% *** Do not adjust lengths that control margins, column widths, etc. ***
% *** Do not use packages that alter fonts (such as pslatex).         ***
% There should be no need to do such things with IEEEtran.cls V1.6 and later.
% (Unless specifically asked to do so by the journal or conference you plan
% to submit to, of course. )


% correct bad hyphenation here
\hyphenation{op-tical net-works semi-conduc-tor}
\renewcommand\IEEEkeywordsname{Keywords}


\begin{document}
%
% paper title
% Titles are generally capitalized except for words such as a, an, and, as,
% at, but, by, for, in, nor, of, on, or, the, to and up, which are usually
% not capitalized unless they are the first or last word of the title.
% Linebreaks \\ can be used within to get better formatting as desired.
% Do not put math or special symbols in the title.
\title{Controle de estabilidade veicular utilizando controle preditivo LTI sem
restrições}%
%
% author names and IEEE memberships
% note positions of commas and nonbreaking spaces ( ~ ) LaTeX will not break
% a structure at a ~ so this keeps an author's name from being broken across
% two lines.
% use \thanks{} to gain access to the first footnote area
% a separate \thanks must be used for each paragraph as LaTeX2e's \thanks
% was not built to handle multiple paragraphs
%


\author{Z. R. Magalhães Jr.,% <-this % stops a space
\thanks{Z. R. Magalhães Jr, Departamento de engenharia mecânica,
Universidade de Brasília(UNB), Brasília, Distrito Federal, Brasil, zoemagalhaes@aluno.unb.br }% <-this % stops a space
\vspace*{-0.82cm}
}
%\author{Zoé Magalhães, André Murilo and Renato Lopes}


% note the % following the last \IEEEmembership and also \thanks - 
% these prevent an unwanted space from occurring between the last author name
% and the end of the author line. i.e., if you had this:
% 
% \author{....lastname \thanks{...} \thanks{...} }
%                     ^------------^------------^----Do not want these spaces!
%
% a space would be appended to the last name and could cause every name on that
% line to be shifted left slightly. This is one of those "LaTeX things". For
% instance, "\textbf{A} \textbf{B}" will typeset as "A B" not "AB". To get
% "AB" then you have to do: "\textbf{A}\textbf{B}"
% \thanks is no different in this regard, so shield the last } of each \thanks
% that ends a line with a % and do not let a space in before the next \thanks.
% Spaces after \IEEEmembership other than the last one are OK (and needed) as
% you are supposed to have spaces between the names. For what it is worth,
% this is a minor point as most people would not even notice if the said evil
% space somehow managed to creep in.



% The paper headers
%\markboth{Journal of \LaTeX\ Class Files,~Vol.~14, No.~8, August~2015}%
%{Shell \MakeLowercase{\textit{et al.}}: Bare Demo of IEEEtran.cls for IEEE
% Journals}
% The only time the second header will appear is for the odd numbered pages
% after the title page when using the twoside option.
% 
% *** Note that you probably will NOT want to include the author's ***
% *** name in the headers of peer review papers.                   ***
% You can use \ifCLASSOPTIONpeerreview for conditional compilation here if
% you desire.




% If you want to put a publisher's ID mark on the page you can do it like
% this:
%\IEEEpubid{0000--0000/00\$00.00~\copyright~2015 IEEE}
% Remember, if you use this you must call \IEEEpubidadjcol in the second
% column for its text to clear the IEEEpubid mark.



% use for special paper notices
%\IEEEspecialpapernotice{(Invited Paper)}




% make the title area
\maketitle
\thispagestyle{empty}

\begin{otherlanguage}{english} 

% As a general rule, do not put math, special symbols or citations
% in the abstract or keywords.
\begin{abstract} 
In this paper, a unconstreined linear model predicitive control  was
employed to vehicle lateral stability control. The control algorithm 
was implemented as an aplication for Linux. Simulation with the control
and the vehicle model running in a RaspberryPi 2011.12 was performed
to validate this implementation. The model used in the control disign and
to simulate  the vehicle dynamic  is a model of the
response of the lateral slip, yaw rate, roll rate and roll angle to the front
wheels steer angle and an external yaw momment, linear and stable in open loop.
The simulation result shows that the control is able to make  
yaw rate to track its reference without take lateral slip to high values.
\end{abstract}

% Note that keywords are not normally used for peerreview papers.
\begin{IEEEkeywords}
Vehicle lateral stability control, hardware in the loop,  linear quadratic
regulator.
\end{IEEEkeywords}
\end{otherlanguage}





% For peer review papers, you can put extra information on the cover
% page as needed:
% \ifCLASSOPTIONpeerreview
% \begin{center} \bfseries EDICS Category: 3-BBND \end{center}
% \fi
%
% For peerreview papers, this IEEEtran command inserts a page break and
% creates the second title. It will be ignored for other modes.
\IEEEpeerreviewmaketitle
 \section{Introdução}
 
\IEEEPARstart{C}{ontroladores} eletrônicos de estabilidade
(\emph{ESC}) são sistemas de assistência ao motorista na correção de movimentos que desviem o veículo da
trajetória desejada, contribuindo para a redução de acidentes causados
pelo deslizamento lateral e capotamento. 
Muitas pesquisas foram realizadas nos últimos anos para propor implemetações de
\emph{ESCs}, em que várias técnicas de controle têm sido aplicadas
\cite{Zhang2016, Zhai2016, Yuan2016,Li2017, Nahidi2017, Jin2017a, Jalali2017, Niaona2017, Guo2017}.
Em \cite{Yogurtcu2015} foram desenvolvidos dois sistemas que aplicam o
\emph{PID} e o \emph{LQR}, simulações do controle do veículo representado pelo
modelo bicicleta foram realizadas para validar estes controladores.
Em \cite{Subroto2017} um controle de modo deslizante adaptativo foi projetado
para aumentar a estabilidade lateral de veículos com tração independente nas quatro rodas.
Em \cite{Zhou2010} a técnica \emph{Backstepping} foi aplicada no projeto
de um controlador da taxa de guinada mediante comando do sistema de freios.

Uma arquitetura comum nestes sistemas é a separação do controle em dois níveis:
o alto nível, que calcula as forças e torques que precisam ser aplicados ao
veículo para estabilizá-lo, e o baixo nível, que
comanda os atuadores para que os valores calculados pelo alto
nível sejam alcançados.
Uma vantagem dessa arquitetura é o potencial
do controlador de alto nível para ser aplicado a veículos com diferentes
sistemas de atuação, mas com a desvatangem de não ser auto suficiente por
depender de um controlador de baixo nível para comandar os atuadores.

Este artigo apresenta um controlador de alto nível que aplica o
controle preditivo baseado em modelo linear sem restrições. Este
controlador foi implementado como uma preparação para um controle preditivo
baseado em modelo não linear que considere as restrições físicas do sinal
de comando e imponha restrições aos estados para mantê-los dentro
das faixas de valores desejados.  

Em \cite{Li2017} e \cite{Nahidi2017} foram propostos
controladores preditivos (\emph{MPC}) para calcular a força exigida de cada
roda, e para a obtenção dessas forças, foi utilizado em \cite{Li2017} um
controlador robusto do sistema de freios diferenciais e em  \cite{Nahidi2017} um
controlador ótimo do sistema de distribuição de torque.
O controlador proposto neste artigo calcula o momento de guinada externo a ser
aplicado ao veículo para corrigir o deslizamento lateral e movimentos de
rolagem indesejados e fazer a taxa de guinada rastrear uma trajetória desejada.

Neste trabalho, o algoritmo de controle foi implementado como uma aplicação para
Linux. Para validar está implementação foram realizadas simulações em que
o algoritmo de controle e o modelo da dinâmica veícula são executados em uma
\emph{RaspberryPi C 2010.2011} com sistema operacional 
\emph{RASPBIAN STRETCH LITE version Appril 2018}.

Este documento está organizado da seguinte forma: a Seção
\ref{sec:model} apresenta o modelo matemático utilizado para simular a dinâmica
veicular; a Seção \ref{sec:control} descreve o sistema de controle proposto;
a Seção \ref{sec:result} apresenta os experimentos de validação do sistema
proposto e a Seção \ref{sec:conc} apresenta a conclusão.
 \section{ Dinâmica veicular }
 \label{sec:model}
 
O modelo da dinâmica veicular utilizado neste trabalho, a partir do qual
se obteve o modelo linear utilizado para projetar o controlador e simular
a dinâmica veícular, é composto dos movimentos lateral, longitudinal e de
guinada e de rolagem \cite{Zheng2006,Miao2015}.
Como pretende-se observar a resposta do veículo ao momento de guinada externo
calculado pelo controlador sem considerar o sistema de atuação utilizado para
isso, o modelo não considera as forças longitudinais geradas pelas rodas, que
dependem do efeito dos atuadores na tração transferida para as rodas
\cite{Shoutao2017}.
A Fig. \ref{fig:modeldraw}  ilustra este modelo represetentado pelas seguintes
equações:

\begin{figure}[t]
    \centering
    \includegraphics[width=\columnwidth]{carmodel.png}
    \caption{ Modelo do veículo com as forças aplicadas a ela e suas componentes
    de movimento}
    \label{fig:modeldraw}
\end{figure}

\noindent\textbf{Movimento longitudinal:}
    \begin{equation}
    \label{eq:model}
    \begin{split}
        m\left(\dot{u} - \dot{\psi} v \right) + m_sh_s\dot{\psi}\dot{\phi} = 
    	-\sin(\delta_f)( F_{yfl} + F_{yfr} )  
    	\\
    \end{split}
    \end{equation}
\noindent\textbf{Movimento lateral:} 
    \begin{equation}
    \label{eq:model}
    \begin{split}
    	m\left(\dot{v} + \dot{\psi} u \right) - m_sh_s\ddot{\phi} =& 
    	\cos(\delta_f)( F_{yfl} + F_{yfr} )\\ +&  F_{yrl} +  F_{yrr} \\
    \end{split}
    \end{equation}
\noindent\textbf{Movimento de guinada:} 
    \begin{equation}
    \label{eq:model}
    \begin{split}
        I_{zz}\ddot{\psi} - I_{xz}\ddot{\phi} = 
          &   a\cos(\delta_f)(F_{yfl} + F_{yfr}) - b(F_{yrl} + F_{yrr}) \\
        - & \frac{t\sin(\delta_f)}{2}(F_{yfr} - F_{yfl} ) \\
        + & M_{u}
    \end{split}
    \end{equation}
\noindent\textbf{Movimento de rolagem:}
\begin{equation}
\label{eq:model}
\begin{split}
    I_{xx}\ddot{\phi} - I_{xz}\ddot{\psi} & = m_sh_s(\dot{v} +
    \dot{\psi}u) + m_sh_sg\sin\phi \\
    &- (k_{\phi f} +  K_{\phi r})\phi - (c_{\phi f} + c_{\phi r})\dot{\phi} \\
 \end{split}        
\end{equation}

\noindent em que $v$ é a velocidade lateral, $u$ é a velocidade longitudinal,
$\psi$ é o ângulo de guinada, $\phi$ é o ângulo de rolagem, $\delta_f$ é o
ângulo de esterçamento das rodas dianteiras, $m$ é a massa total do veículo, 
$m_s$ é a massa do veículo sobre o eixo de rolagem, $h_s$ é a distância entre o
centro de rolagem e o centro de massa, $I_{zz}$ e $I_{xx}$ são os momentos de inércia de guinada e rolagem, respectivamente,
$I_{xz}$ é o produto de inércia em relação aos eixos de guinada e de rolagem,
$M_u$ é o momento externo de guinada calculado e
$F_{yfl},F_{yr},F_{yrl},F_{yrr}$ são as forças latereais exercidas pelas rodas.

\vspace{-5mm}
\subsection{ Dinâmica dos pneus }

As forças geradas pelas rodas são governadas por relações complexas entre o
atrito pneu-pista, carga, ângulo de deslizamento e propriedades dos pneus. Os
modelos amplamente utilizados baseiam-se em formulações empíricas derivadas de
testes experimentais \cite{Dugoff1969}. 
As não linearidades na dinâmica dos pneus 
foram consideradas neste trabalho pela utilização da \emph{Magic Formula} (\emph{MF}) 
para calcular a força exercida por cada roda \cite{Shoutao2017}.
Na \emph{MF} a força lateral $F_{yi}$ exercida pela roda é modelada em função de
carga $F_{zi}$ e o ângulo de deslizamento lateral $\alpha_i$ da roda.
A formulação geral da \emph{MF} é expressa por \cite{Li2017}:
\vspace{-2.5mm}
\begin{equation}
\label{eq:MFY}
\begin{split}
		F_{wyi}  & = D\sin(C\arctan[B(1-E)\alpha_i \\ 
			     & + E\arctan(B\alpha_i)]) \\
			  C  & = a_0 \\
			  D  & = F_{zi}(a_1F_{zi} + a_2 ) \\
			  B  & = \frac{(a_3\sin(2\arctan\frac{F_{zi}}{a_4})}{CD} \\
			  E  & = a_6F_{zi} + a_7 \\
			  i  & = fl,fr,rl,rr
\end{split}
\end{equation}

\noindent em que $\gamma$ é o ângulo entre o eixo de rotação da roda e a pista,
normalmente assumido como zero 

Considerando a aceleração longitudinal $a_x$ e lateral $a_y$, a carga individual
de cada roda pode ser calculada por:
\begin{equation}
\label{eq:wload}
\begin{split}
    F_{zfl} =  \frac{mgb}{2l} - \frac{ma_xh}{2l} - \frac{ma_yah}{lt_f} -
    \frac{k_{\phi f}\phi}{t_f} - \frac{c_{\phi f}\dot\phi}{t_f} \\
    F_{zfr} =  \frac{mgb}{2l} - \frac{ma_xh}{2l} + \frac{ma_yah}{lt_f} +
    \frac{k_{\phi f}\phi}{t_f} + \frac{c_{\phi f}\dot\phi}{t_f} \\
    F_{zrl} =  \frac{mga}{2l} + \frac{ma_xh}{2l} - \frac{ma_yah}{lt_r} - 
    \frac{k_{\phi r}\phi}{t_r} - \frac{c_{\phi r}\dot\phi}{t_r} \\
    F_{zrr} =  \frac{mga}{2l} + \frac{ma_xh}{2l} + \frac{ma_yah}{lt_r} + 
    \frac{k_{\phi r}\phi}{t_r} + \frac{c_{\phi r}\dot\phi}{t_r}
\end{split}
\end{equation}
\noindent em que $a$ é a distância das rodas dianteiras do centro de massa, $b$
é a distância das rodas traseiras do centro de massa, $h$ é a altura do veículo,
$l$ é a distância entre os eixos dianteiro e traseiro, $t_f$ e $t_r$ são as
larguras dos eixos dianteira e traseiro, respectivamente.

O ângulo de deslizamento lateral de cada roda, definido como o ângulo entre o
vetor velocidade linear da roda e o seu eixo longitudinal, pode ser calculado
por:
\begin{equation}
	\label{eq:alphaall}
	\begin{split}
		\alpha_{fl} = \delta_f - \arctan\left(\frac{v + a\dot{\psi}}{u -
		\frac{t_f}{2}\dot{\psi}}\right)  & \enskip \alpha_{rl} = -
		\arctan\left(\frac{v - b\dot{\psi}}{u - \frac{t_r}{2}\dot{\psi}}\right)\\
		\alpha_{fr}  = \delta_f - \arctan\left(\frac{v + a\dot{\psi}}{u +
		\frac{t_f}{2}\dot{\psi}}\right)  & \enskip  \alpha_{rr} = - \arctan\left(\frac{v - b\dot{\psi}}{u +
        \frac{t_r}{2}\dot{\psi}}\right)\\
	\end{split}
\end{equation}

\section{Sistema de controle}
\label{sec:control}
    
O sistema proposto aplica o controle preditivo baseado em modelo linear
sem restrições para calcular o momento de guinada externo em função dos erros
da taxa de guinada, do ângulo de deslizamento lateral, da taxa de rolagem
e do ângulo de rolagem.

Considera-se que o controlador tem acesso aos quatro estados do modelo linear.
Como os movimentos de rolagem e o deslizamento lateral são movimentos
indesejados, a trajetória desejada para estes estados é constante e igual a
zero.
O valor de referência assumido para a taxa de guinada é calculado conforme 
descrito em \cite{Jin2017,Zheng2006}:
\begin{equation}
	\label{eq:desiredyaw}
	\begin{split}
		\dot{\psi}_d  = \frac{u\delta_f}{l+lK_uu^2} \enskip&\enskip 
		K_u = \frac{m}{l^2}\left(\frac{b}{C_r} - \frac{a}{C_f}\right)
	\end{split}
\end{equation}
\begin{equation}
	\begin{split}
		|\dot{\psi}_d| \leq \left|\frac{\mu g}{u}\right| 
	\end{split}
\end{equation}

O  momento de guinada é calculado pela lei de controle :

$M_u(k)= K_Nx(k) + G_N\tilde{y}_d + L_Nu_d $,
\noindent em que as matrizes matrizes $K_N$, $G_N$, $L_N$ são obtidos
conforme descrito em \cite{alamir2013pragmatic} como solução ótima da função
custo sem restrição do controle preditivo para sistemas LTI, $x$ é o vetor de
estados, $\tilde{y}_d$ é um vetor composto da concatenação dos vetores de
estados que formam a trajetória desejada no hirizonte de predição e $u_d$ é 
o vetor de comando desejado para o estado estacionário, assumido como zero
porque considera-se que as curvas do veículo são transitórias, pois o veículo
sempre retorna para a condução em linha reta quando nenhum momomento externo
deve ser aplicado.

O modelo LTI utilizado no projeto do \emph{MPC}} consiste de uma
linearização do modelo apresentado para a dinâmica veicular, assumindo
velocidade longitudinal constante igual a 80km/h e desprezando a equação
do movimento longitudinal. O resultado é um modelo composto pelos movimentos
lateral, de guinada e de rolagem representados pelas equações:
\textbf{Movimento lateral:}
\begin{equation}
\label{eq:linearlat}
\begin{split}
	m\left(\dot{v} + \dot{\psi} u \right) -m_sh_s\ddot{\phi}= \sum{F_y} \\
\end{split}
\end{equation}
\textbf{Movimento de guinada:} 
\begin{equation}
\label{eq:lineargui}
\begin{split}
	I_{zz}\ddot{\psi} - I_{xz}\ddot{\phi} = a(F_{yfl} + F_{yfr}) - b(F_{yrl} +
	F_{rr}) + M_u \\
\end{split}
\end{equation}
\textbf{Movimento de rolagem:}
\begin{equation}
\label{eq:linearrol}
\begin{split}
	I_{xx}\ddot{\phi} - I_{xz}\ddot{\psi} = & m_sh_s(\dot{v} + \dot{\psi}u) +
	m_sh_sg\phi - \\&(k_{\phi f} + k_{\phi r} ) \phi - ( c_{\phi f} + c_{\phi r} )
	\dot{\phi} \\
\end{split}
\end{equation}

Para as forças dos pneus é utilizada a aproximação linear da \emph{MF}:
\begin{equation}
	\label{eq:FMF}
\begin{split}
	F_{yfi} \approx C_f\alpha_f \enskip&\enskip F_{yri} \approx C_r\alpha_r
\end{split}
\end{equation} 
\noindent em que $C_r$ e $C_f$ são denominados coeficientes de rigidez nas curvas das rodas traseiras e dianteiras, 
respectivamente, os seus valores são obtidos por expansão de Taylor:
\begin{equation}
\begin{split}
		C_f = \left.\frac{\partial F_{yi}}{\partial \alpha_i}\right|_{x_e} \enskip&\enskip
		C_r = \left.\frac{\partial F_{yi}}{\partial \alpha_i}\right|_{x_e}\\
	i = fl,fr \enskip&\enskip i = rl,rr
	\end{split}
\end{equation}
\noindent em que $x_e$ representa o ponto de linearização.

O ângulo de deslizamento das rodas aproximado é calculado em função do deslizamento lateral $\beta$, 
da taxa de rolagem $\dot{\psi}$ e do esterçamento das rodas dianteiras $\delta_f$:
\begin{equation}
	\label{eq:alphalin}
\begin{split}
		\alpha_f  = -\beta - \frac{a\dot{\psi}}{u} + \delta_f \enskip&\enskip
		\alpha_r  = -\beta + \frac{b\dot{\psi}}{u}
	\end{split}
\end{equation}
\begin{equation}
	\label{eq:beta}
	\beta =\arctan\left(\frac{v}{u}\right)
\end{equation}	

Considerando a velocidade longitudinal constante, a manipulação algébrica das
Equações \ref{eq:linearlat} - \ref{eq:alphalin} permite obter uma representação
do modelo linear em espaço de estados: $\dot{x} = Ax + BM_u+ E\delta_f$. 
\section{Resultados}
\label{sec:result}

Uma biblioteca para c++ batizada de libmpc foi desenvolvida neste trabalho para 
disponibilizar métodos para implementação de MPC para sistemas LTI sem
restrições. Para as operações aritméticas entre matrizes e vetores 
foi utilizada a biblioteca Eigen v3.3.4 disponibilizada em \cite{eigenweb}.
Esta biblioteca foi utilizada para implementar o controlador de estabilidade
lateral proposto. 

No projeto do controlador as ponderações atribuida aos erros dos estados foram
$Q_Y$ = [4,1,1E-4,1E-3] e para o erro no comando foi atribuída uma ponderação de
1E-12. O horizonte de predição adotado foi de 100 amostras, que correspondem a
8ms, pois o período de amostragem adotado foi 8e-5.

Simulações foram realizadas para validar se o controlador é capaz de fazer a 
taxa de guinada rastrear uma trajetória desejado enquanto o ângulos de deslizamento
lateral e rolagem permanecem pequenos (<10 graus) de forma a manter o sistema em
uma zona de operação próxima ao ponto de linearização. Nestas simulações o 
controlador e o modelo computacional da dinâmica do veículo foram integrados em
uma aplicação executada em RaspberryPi 2011.12 com sistema operacioal \emph{RaspberryPi C 2010.2011} com sistema operacional 
\emph{RASPBIAN STRETCH LITE version Appril 2018}. Para simulação da dinâmica
veicular foi utilizado o modelo linear apresentado em \ref{sec:control}.
Os parâmetros do modelo estão apresentados na tabela \ref{tab:model}.

As manobras escolhidas para os testes foram a anzol e a mudança dupla de faixa
em alta velocidade e com alto esterçamento das rodas dianteiras. 
A manobra anzol consiste de uma curva breve em uma direção seguida de
uma curva de longa duração na direção oposta \cite{Mashadi2010,Dahmani2015,Dahmani2016,Li2017}.
A mudança dupla são duas curvas breves em direções
opostas com um curto intervalo entre elas \cite{Ren2015,Rafaila2016,Zhai2016}.
Essas manobras foram testadas com velocidade inicial de 80km/h e esterçamento
máximo da rodas dianteiras igual a 10 graus.

\begin{figure*}[h]
    \centering
    \includegraphics[width=\textwidth]{fish.jpg}
    \caption{ Resultados obtidos para a manobra anzol}
    \label{fig:testeFish}
\end{figure*}
\vskip
\begin{figure*}[h!]
    \centering
    \includegraphics[width=\textwidth]{doublelane.jpg}
    \caption{ Resultados obtidos para a mudança de faixa dupla }
    \label{fig:testeDouble}
\end{figure*}

Os resultados mostram que nas duas manobras o controlador foi capaz de fazer a
taxa de guinada seguir a trajetória desejada com valor superior ao seu valor em
malha aberta e ainda diminui o deslizamento lateral e os movimentos de rolagem.

\begin{table}[t!]
\caption{ Parâmetros do modelo utilizados na simulação}
\label{tab:model}
\begin{adjustbox}{width=\columnwidth,center}
\begin{tabular}{|ll|ll|}
\hline
  Parâmetro     & Valor           & Parâmetro       & Valor \\
 \hline
  a             & 1.035 m          &$a_0$             & 1.6   \\
  b             & 1.655 m          &$a_1$             & -34  \\
  $t_f$         & 1.520 m          &$a_2$             & 1250 \\
  $t_r$         & 1.540 m          &$a_3$             & 2320 \\
  hs            & 0.451 m          &$a_5$             & 0  \\
  hg            & 0.550 m          &$a_6$             & -0.0053 \\
  m             & 1704.7 Kg        &$a_7$             & 0.1925  \\
  $m_s$         & 1526.9 Kg        & $c_{\phi f}$     & 2823N m s/rad \\
  $I_{zz}$      & 3048.1 $Kgm^2$   & $c_{\phi r}$     & 2652N m s/rad\\
  $c_{\phi r}$  & 2652 Nms/rad     & $k_{\phi f}$     & 47298N m /rad\\
  $I_{xx}$      & 744.0 $Kgm^2$    & $k_{\phi r}$     & 47311N m /rad\\
  $I_{xz}$      & 50 $Kgm^2$       &                  & \\
\hline
\end{tabular}
\end{adjustbox}
\end{table}
  
\section{ Conclusão }
\label{sec:conc}

Este artigo apresentou o projeto de um controlador preditivo para 
sistemas LTI sem restrições. O controlador proposto foi implementado em uma
RaspberryPi 2011.12.

Simulações foram realizadas para validar se o controlador proposto é capaz de
fazer a taxa de guinada rastrear uma trajetória de referencia e corrigir os
movimentos de guinada e rolagem. Os cenários de testes observados foram 
as manobras anzol e mudança dupla de faixa, ambas com esterçamento máximo
das rodas dianteiras igual a 10 graus.

Os resultados mostram que o controlador foi capaz de atingir o objetivo
esperado.

Este trabalho foi feito como etapa de preparação para implementação de um
controlador preditivo para sistemas não lineares com restrições.

\bibliographystyle{IEEEtran}
\bibliography{mylib}

\vspace*{-2.5\baselineskip} 
\begin{IEEEbiography}
[{\includegraphics[width=1in,height=1.25in,clip,keepaspectratio]{zoe.jpg}}]{Zoé
Roberto Magalhães Júnior} possui graduação em Engenharia Eletrônica pela
Universidade de Brasília (2014),  onde cursa desde 2017 o mestrado em Engenharia
Mecatrônica. Possui experiência no desenvolvimento de firmware para
microcontroladores e sistemas embarcados de produtos eletrônicos automotivos. E
atualmente realiza pesquisas sobre controle de estabilidade veicular.
\end{IEEEbiography}
%{

\end{document}    
  
